\subsection{Topological Sort (1-indexed)}
\begin{cpp}
struct Graph{
  int n; vector<vector<int>> adj;
  Graph(int n): n{n}, adj(n+1) {}
  void connect(int a, int b){adj[a].push_back(b);}
  vector<int> toposort(){
    vector<int> indg(n+1), L;
    queue<int> Q;
    for(int i=1; i<=n; i++) for(int j: adj[i]) indg[j]++;
    for(int i=1; i<=n; i++) if(!indg[i]) Q.push(i);
    for(int i=0; i<n; i++){
      if(Q.empty()) assert(0); // Change this if you need
      int p = Q.front(); Q.pop();
      L.push_back(p);
      for(int j: adj[p]) if (!--indg[j]) Q.push(j);
    }
    return L;
  }
};
\end{cpp}

\subsection{SCC (1-indexed)}
\begin{cpp}
struct Graph{
  int n, cnt, sccnt; vector<vector<int>> adj;
  vector<int> up, visit, scx, stk;
  Graph(int n): n(n), adj(n+1), up(n+1), visit(n+1), scx(n+1) {}
  void connect(int a, int b){adj[a].push_back(b);}
  void dfs(int v){
    up[v] = visit[v] = ++cnt;
    stk.push_back(v);
    for(int nxt: adj[v]){
      if(!visit[nxt]) dfs(nxt), up[v] = min(up[v], up[nxt]);
      else if(!scx[nxt]) up[v] = min(up[v], visit[nxt]);
    }
    if(up[v] == visit[v]){
      ++sccnt; int t = -1;
      while(!stk.empty() && t != v){
        t = stk.back(); stk.pop_back();
        scx[t] = sccnt;
      }
    }
  }
  void getscc(){
    cnt = sccnt = 0;
    for(int i=1; i<=n; i++) if(!visit[i]) dfs(i);
  }
  vector<vector<int>> scc(){
    getscc();
    vector<vector<int>> res(sccnt);
    for(int i=1; i<=n; i++) res[scx[i]-1].push_back(i);
    return res;
  }
  vector<vector<int>> dag(){
    getscc();
    vector<vector<int>> res(*max_element(entire(scx))+1);
    for(int v=1; v<=n; v++){
      for(int u: adj[v]) if(scx[u] != scx[v]) 
        res[scx[v]].push_back(scx[u]);
    }
    return res;
  }
  vector<vector<int>> rdag(){
    getscc();
    vector<vector<int>> res(*max_element(entire(scx))+1);
    for(int v=1; v<=n; v++){
      for(int u: adj[v]) if(scx[u] != scx[v]) 
        res[scx[u]].push_back(scx[v]);
    }
    return res;
  }
};
\end{cpp}

\subsection{2-SAT}
\begin{cpp}
struct TwoSAT{
  int n; Graph G;
  TwoSAT(int var): n{var}, G{Graph(2*var+1)} {}
  int v(int x){return n+x+1;}
  void cnf(int x, int y)
    {G.connect(v(-x),v(y)); G.connect(v(-y),v(x));}
  bool solve(){
    G.getscc();
    for(int x=1; x<=n; x++) if(G.scx[v(x)] == G.scx[v(-x)])
      return false;
    return true;
  }
};

struct TwoSATsol{
  vector<int> solve(){
    vector<vector<int>> scc = G.scc();
    vector<int> sol(n+1, -1);
    for(int x=1; x<=n; x++)
      if(G.scx[v(x)] == G.scx[v(-x)]) return {0};
    for(int i = scc.size()-1; i>=0; i--)
      for(int var: scc[i]) if(sol[abs(var-n-1)] == -1)
        sol[abs(var-n-1)] = var<n+1;
    return sol;
  }
};
\end{cpp}

\subsection{Stoer-Wagner (0-indexed)}
\begin{cpp}
struct Mincut{
  int n; vector<vector<int>> graph;
  Mincut(int n): n{n}, graph(n, vector<int>(n)) {}
  void connect(int a, int b, int w)
    {if(a != b) graph[a][b]+= w, graph[b][a]+= w;}
  
  pair<int, pair<int, int>> stmin(vector<int> &active){
    vector<int> key(n), v(n);
    int s = -1, t = -1;
    for(size_t i=0; i<active.size(); i++){
      int maxv = -1, cur = -1;
      for(auto j: active) if(!v[j] && maxv < key[j])
        maxv = key[j], cur = j;
      t = s, s = cur; v[cur] = 1;
      for(auto j: active) key[j]+= graph[cur][j];
    }
    return make_pair(key[s], make_pair(s, t));
  }

  vector<int> cut;
  int solve(){
    int res = numeric_limits<int>::max();
    vector<vector<int>> grps; vector<int> active;
    cut.resize(n);
    for(int i=0; i<n; i++) grps.emplace_back(1, i);
    for(int i=0; i<n; i++) active.push_back(i);
    while(active.size() >= 2){
      auto stcut = stmin(active);
      if(stcut.first < res){
        res = stcut.first;
        fill(entire(cut), 0);
        for(auto v: grps[stcut.second.first]) cut[v] = 1;
      }
      int s, t; tie(s, t) = stcut.second;
      if(grps[s].size() < grps[t].size()) swap(s, t);
      active.erase(find(entire(active), t));
      grps[s].insert(grps[s].end(), entire(grps[t]));
      for(int i=0; i<n; i++)
        graph[i][s]+= graph[i][t], graph[i][t] = 0;
      for(int i=0; i<n; i++)
        graph[s][i]+= graph[t][i], graph[t][i] = 0;
      graph[s][s] = 0;
    }
    return res;
  }
};
\end{cpp}

\subsection{Offline Dynamic Connectivity}
\begin{cpp}
struct DisjointHist{
  vector<int> par, sz;
  vector<pair<int, int>> history;
  DisjointHist(int n): par(n+1), sz(n+1, 1)
    {iota(entire(par), 0);}

  void un(int x, int y){
    x = fd(x), y = fd(y);
    if(x == y) return;
    if(sz[x] < sz[y]) swap(x, y);
    history.push_back({x, y});
    par[y] = x, sz[x]+= sz[y];
  }
  int fd(int x){return par[x]!=x? fd(par[x]) : x;}
  void rollback(){
    auto [x,y] = history.back(); history.pop_back();
    par[x] = x, par[y] = y, sz[x]-= sz[y];
  }
};

struct ODC{
  int n, Q, time = 0;
  DisjointHist DS;
  vector<map<int,int>> adj; // [u][v] -> addtime
  vector<int> end; // ending time of edge added at i
  vector<int> qtype; // 1 add 2 remove 3 question
  vector<pair<int,int>> quest; // pair of v at query i
  vector<bool> answer; // answer to the questions
  ODC(int n, int Q): n{n}, Q{Q}, DS(n), adj(n+1),
    end(Q), qtype(Q), quest(Q), answer(Q) {}

  void add(int x, int y){
    if(x > y) swap(x, y);
    quest[time] = {x,y}, qtype[time] = 1, end[time] = Q-1;
    adj[x][y] = time++;
  }
  void rem(int x, int y){
    if(x > y) swap(x, y);
    quest[time] = {x,y}, qtype[time] = 2;
    end[adj[x][y]] = time++;
  }
  void query(int x, int y){
    if(x > y) swap(x, y);
    quest[time] = {x, y}, qtype[time++] = 3;
  }

  void solve(){
    vector<int> E;
    for(int i=0; i<Q; i++) if(qtype[i] == 1) 
      E.push_back(i);
    solve(0, Q-1, E);
  };
  void solve(int s, int e, vector<int> &E){
    if(s > e) return;
    int startnum = DS.history.size();

    int mid = (s+e)/2;
    vector<int> E1, E2;
    for(int i: E){
      auto [x,y] = quest[i];
      if(i <= s && e <= end[i]) DS.un(x, y);
      else{
        if(!(end[i] < s || i > mid)) E1.push_back(i);
        if(!(end[i] < mid+1 || i > e)) E2.push_back(i);
      }
    }

    if(s == e){
      auto [x,y] = quest[s];
      if(qtype[s] == 3) answer[s] = (DS.fd(x)==DS.fd(y));
    }
    else solve(s, mid, E1), solve(mid+1, e, E2);
    while(DS.history.size() != startnum) DS.rollback();
  }
};
\end{cpp}