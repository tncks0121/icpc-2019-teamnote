\subsection{Arithmetic stuff}
\begin{cpp}
//ceil(a/b)
ll ceildiv(ll a, ll b){
  if(b < 0) return ceildiv(-a, -b);
  if(a < 0) return (-a) / b;
  return ((ull)a + (ull)b - 1ull) / b;
}

//floor(a/b)
ll floordiv(ll a, ll b){
  if(b < 0) return floordiv(-a, -b);
  if(a >= 0) return a / b;
  return -(ll)(((ull)(-a) + b - 1) / b);
}

ll __gcd(ll a, ll b);

//egcd (ac + bd = gcd(a, b))
pair<ll, ll> egcd(ll a, ll b){
  if(b == 0) return {1, 0};
  auto t = egcd(b, a%b);
  return {t.second, t.first-t.second*(a/b)};
}

//modular inverse (ax = gcd(a, m) mod m)
ll modinv(ll a, ll m){return (egcd(a,m).first%m + m)%m;}

//modular power
ll pow(ll a, ll x, ll mod){
  a%= mod; ll r;
  for(r=1; x; x/=2)
    {if(x%2) r=modmul(r,a,m); a=modmul(a,a,m);}
  return r;
}

//modular large multiplication
ll modmul(ll a, ll b, ll m) {
  return ll((__int128)a*(__int128)b%m);
}

int __builtin_clz(int x); // number of leading zero
int __builtin_ctz(int x); // number of trailing zero
int __builtin_popcount(int x); // number of 1-bits in x
// use clzll, ctzll, popcountll for long long x

last_bit == n&-n;
floor(log2(n)) == 31 - __builtin_clz(n|1);
floor(log2(n)) == 63 - __builtin_clzll(n|1);

// 00111 01011 01101 01110 10011 ...
ll next_perm(ll v){
    ll t = v|(v-1);
    int c = __builtin_ctz(v);
    return (t+1) | (((~t & -~t)-1) >> (c+1));
}
\end{cpp}

\subsection{Primality}
\begin{cpp}
typedef unsigned long long ull;
bool witness(ull a, ull n, ull s){
  if(a >= n) a%= n;
  if(a <= 1) return true;
  ull d = n>>s, x = pow(a, d, n);
  if(x == 1 || x == n-1) return true;
  while(s-- > 1){
    x = modmul(x, x, n);
    if(x == 1) return false;
    if(x == n-1) return true;
  }
  return false;
}

bool millerRabin(ull n){
  if(n == 2) return true;
  if(n < 2 || n%2 == 0) return false;
  ull d = n>>1, s = 1;
  for(; (d&1)==0; s++) d>>=1;
#define T(a) witness(a##ull, n, s)
  if(n < 4759123141ull) return T(2) && T(7) && T(61);
  return T(2) && T(325) && T(9375) && T(28178)
    && T(450775) && T(9780504) && T(1795265022);
}

ll rho(ll n){
  random_device rd; mt19937 gen(rd());
  uniform_int_distribution<ll> dis(1, n-1);
  ll x = dis(gen), c = dis(gen), g = 1; ll y = x;
  while(g == 1){
    x = (modmul(x, x, n) + c) % n;
    y = (modmul(y, y, n) + c) % n;
    y = (modmul(y, y, n) + c) % n;
    g = __gcd(abs(x-y), n);
  }
  return g;
}

void factorize(ll n, vector<ll>& fl){
  if(n == 1) return;
  if(n%2 == 0) fl.push_back(2), factorize(n/2, fl);
  else if(millerRabin(n)) fl.push_back(n);
  else{
    ll f = rho(n);
    factorize(f, fl); factorize(n/f, fl);
  }
}
\end{cpp}

\subsection{CRT}
\begin{cpp}
// a = remainders; n = coprime divisors
ll CRT(vector<ll> &a, vector<ll> &n, int pt = 0){
    if(pt == a.size()-1) return a[pt];
    ll a0 = a[pt], a1 = a[pt+1], n0 = n[pt], n1 = n[pt+1];
    ll tmp = modinv(n0, n1);
    ll tmp2 = (tmp*(a1-a0) % n1 + n1) % n1;
    ll ora = a1, tgcd = __gcd(n0, n1);
    a[pt+1] = a0 + n0/tgcd*tmp2;
    n[pt+1]*= n0/tgcd;
    ll ret = CRT(a, n, pt+1);
    n[pt+1]/= n0/tgcd;
    a[pt+1] = ora;
    return ret;
}
\end{cpp}

n들이 서로소가 아니면 소인수분해한 다음 prime power들끼리 남겨 놓고 모순 있는지 판별, 없으면 중복 없애고 CRT 돌리면 됨

\subsection{Binomial Number}
\begin{cpp}
ll nck(ll n, ll k, ll m){
  if(n < k) return 0;
  ll res = 1;
  for(int i=0; i<k; i++){
    res = res*(n-i) % m;
    res = res*pow(i+1, m-2, m) % m;
  }
  return res;
}

ll lucas(ll n, ll k, ll m){
  ll res = 1;
  while(n || k){
    res = res*nck(n%m, k%m, m) % m;
    n/= m, k/= m;
  }
  return res;
}

// or... precalculation and O(1)
// define SZ yourself
ll pow2[SZ+1], fact[SZ+1], ifact[SZ+1];
ll ncr(int n, int k, ll MOD){
    ll a = fact[n]*ifact[k] % MOD;
    return a*ifact[n-k] % MOD;
}
int main(){
  pow2[0] = 1, fact[0] = 1;
  for(int i=1; i<=SZ; i++){
    pow2[i] = pow2[i-1]*2 % MOD;
    fact[i] = fact[i-1]*i % MOD;
  }
  ifact[SZ] = pow(fact[SZ], MOD-2, MOD);
  for(int i=SZ-1; i>=0; i--)
    ifact[i] = ifact[i+1]*(i+1)%MOD;
}
\end{cpp}

\subsection{Matrix}
\begin{cpp}
template <typename T>
struct Matrix{
  int n; T mod; vector<vector<T>> mat;
  Matrix(int n) :n{n}, mat(n, vector<T>(n)) {}
  Matrix operator*(Matrix& B){
    Matrix ans = Matrix(n);
    for(int i=0;i<n;i++) for(int j=0;j<n;j++)
      for(int b=0;b<n;b++)
        ans.mat[i][j]+= mat[i][b]*B.mat[b][j];
    return ans;
  }
  Matrix operator%(T mod){
    Matrix ans = Matrix(n);
    for(int i=0; i<n; i++) for(int j=0; j<n; j++)
      ans.mat[i][j] = mat[i][j]%mod;
    return ans;
  }
};

template <class T>
Matrix<T> matpow(Matrix<T> A, ll x, T mod){
  A = A%mod; Matrix<T> r(A.n);
  for(int i=0; i<A.n; i++) r.mat[i][i] = 1;
  for(; x; x/=2){if(x%2) r = r*A%mod; A = A*A%mod;}
  return r;
}
\end{cpp}

\subsection{FFT}
\begin{cpp}
// f=0 forward; f=1 backward
// 2^17 = 131072; 2^20 = 1048576
int SZ = 19, N = 1<<SZ;
int Rev(int x){
  int r = 0;
  for(int i=0; i<SZ; i++) r = r<<1 | (x&1), x>>= 1;
  return r;
}

const double PI = acos(-1);
typedef complex<double> CD;
typedef vector<CD> VCD;
void FFT(VCD& A, int f){
  CD x, y, z;
  for(int i=0; i<N; i++){
    int j = Rev(i);
    if(i < j) swap(A[i], A[j]);
  }
  for(int i=1; i<N; i<<= 1){
    double w = PI / i;
    if(f) w = -w;
    CD x(cos(w), sin(w));
    for(int j=0; j<N; j+= i<<1){
      CD y(1);
      for(int k=0; k<i; k++){
        CD z = A[i+j+k] * y;
        A[i+j+k] = A[j+k] - z, A[j+k]+= z;
        y*= x;
      }
    }
  }
  if(f) for(int i=0; i<N; i++) A[i]/= N;
}
\end{cpp}

\subsection{More Precise Multiplication With FFT}

TODO

\subsection{NTT}
TODO

\subsection{Gaussian Elimination}
\begin{cpp}
// INPUT:  a[][] = an n*n matrix
//         b[][] = an n*m matrix
// OUTPUT:   X  = an n*m matrix (stored in b[][])
//       A^{-1} = an n*n matrix (stored in a[][])
const double EPS = 1e-10;
typedef vector<vector<double>> VVD;
bool gauss_jordan(VVD& a, VVD& b) {
  const int n = a.size(), m = b[0].size();
  vector<int> irow(n), icol(n), ipiv(n);
  for (int i=0; i<n; i++){
    int pj = -1, pk = -1;
    for(int j=0; j<n; j++) if(!ipiv[j])
      for(int k=0; k<n; k++) if(!ipiv[k])
        if(pj==-1 || fabs(a[j][k]) > fabs(a[pj][pk]))
          pj=j, pk=k;
    if(fabs(a[pj][pk]) < EPS) return false;
    ipiv[pk]++;
    swap(a[pj], a[pk]); swap(b[pj], b[pk]);
    irow[i] = pj; icol[i] = pk;

    double c = 1.0 / a[pk][pk];
    a[pk][pk] = 1.0;
    for(int p=0; p<n; p++) a[pk][p] *= c;
    for(int p=0; p<m; p++) b[pk][p] *= c;
    for(int p=0; p<n; p++) if (p != pk) {
      c = a[p][pk]; a[p][pk] = 0;
      for(int q=0; q<n; q++) a[p][q] -= a[pk][q] * c;
      for(int q=0; q<m; q++) b[p][q] -= b[pk][q] * c;
    }
  }
  for(int p=n-1; p>=0; p--) if(irow[p] != icol[p]){
    for(int k=0; k<n; k++)
      swap(a[k][irow[p]], a[k][icol[p]]);
  }
  return true;
}
\end{cpp}

TODO mod $p$

TODO binary (bitset)

\subsection{Black Box Linear Algebra}

\begin{cpp}
// https://koosaga.com/231
// usage: guess_nth_term(vec, n)

const int mod = 998244353;
using lint = long long;
lint ipow(lint x, lint p){
  lint ret = 1, piv = x;
  while(p){
    if(p & 1) ret = ret * piv % mod;
    piv = piv * piv % mod;
    p >>= 1;
  }
  return ret;
}
vector<int> berlekamp_massey(vector<int> x){
  vector<int> ls, cur;
  int lf, ld;
  for(int i=0; i<x.size(); i++){
    lint t = 0;
    for(int j=0; j<cur.size(); j++){
      t = (t + 1ll * x[i-j-1] * cur[j]) % mod;
    }
    if((t - x[i]) % mod == 0) continue;
    if(cur.empty()){
      cur.resize(i+1);
      lf = i;
      ld = (t - x[i]) % mod;
      continue;
    }
    lint k = -(x[i] - t) * ipow(ld, mod - 2) % mod;
    vector<int> c(i-lf-1);
    c.push_back(k);
    for(auto &j : ls) c.push_back(-j * k % mod);
    if(c.size() < cur.size()) c.resize(cur.size());
    for(int j=0; j<cur.size(); j++){
      c[j] = (c[j] + cur[j]) % mod;
    }
    if(i-lf+(int)ls.size()>=(int)cur.size()){
      tie(ls, lf, ld) = make_tuple(cur, i, (t - x[i]) % mod);
    }
    cur = c;
  }
  for(auto &i : cur) i = (i % mod + mod) % mod;
  return cur;
}
int get_nth(vector<int> rec, vector<int> dp, lint n){
  int m = rec.size();
  vector<int> s(m), t(m);
  s[0] = 1;
  if(m != 1) t[1] = 1;
  else t[0] = rec[0];
  auto mul = [&rec](vector<int> v, vector<int> w){
    int m = v.size();
    vector<int> t(2 * m);
    for(int j=0; j<m; j++){
      for(int k=0; k<m; k++){
        t[j+k] += 1ll * v[j] * w[k] % mod;
        if(t[j+k] >= mod) t[j+k] -= mod;
      }
    }
    for(int j=2*m-1; j>=m; j--){
      for(int k=1; k<=m; k++){
        t[j-k] += 1ll * t[j] * rec[k-1] % mod;
        if(t[j-k] >= mod) t[j-k] -= mod;
      }
    }
    t.resize(m);
    return t;
  };
  while(n){
    if(n & 1) s = mul(s, t);
    t = mul(t, t);
    n >>= 1;
  }
  lint ret = 0;
  for(int i=0; i<m; i++) ret += 1ll * s[i] * dp[i] % mod;
  return ret % mod;
}
int guess_nth_term(vector<int> x, lint n){
  if(n < x.size()) return x[n];
  vector<int> v = berlekamp_massey(x);
  if(v.empty()) return 0;
  return get_nth(v, x, n);
}
struct elem{int x, y, v;};
  // A_(x, y) <- v, 0-based. no duplicate please..
vector<int> get_min_poly(int n, vector<elem> M){
  // smallest poly P such that
  // A^i = sum_{j < i} {A^j \times P_j}
  vector<int> rnd1, rnd2;
  mt19937 rng(0x14004);
  auto randint = [&rng](int lb, int ub){
    return uniform_int_distribution<int>(lb, ub)(rng);
  };
  for(int i=0; i<n; i++){
    rnd1.push_back(randint(1, mod - 1));
    rnd2.push_back(randint(1, mod - 1));
  }
  vector<int> gobs;
  for(int i=0; i<2*n+2; i++){
    int tmp = 0;
    for(int j=0; j<n; j++){
      tmp += 1ll * rnd2[j] * rnd1[j] % mod;
      if(tmp >= mod) tmp -= mod;
    }
    gobs.push_back(tmp);
    vector<int> nxt(n);
    for(auto &i : M){
      nxt[i.x] += 1ll * i.v * rnd1[i.y] % mod;
      if(nxt[i.x] >= mod) nxt[i.x] -= mod;
    }
    rnd1 = nxt;
  }
  auto sol = berlekamp_massey(gobs);
  reverse(sol.begin(), sol.end());
  return sol;
}
lint det(int n, vector<elem> M){
  vector<int> rnd;
  mt19937 rng(0x14004);
  auto randint = [&rng](int lb, int ub){
    return uniform_int_distribution<int>(lb, ub)(rng);
  };
  for(int i=0; i<n; i++) rnd.push_back(randint(1, mod-1));
  for(auto &i : M){
    i.v = 1ll * i.v * rnd[i.y] % mod;
  }
  auto sol = get_min_poly(n, M)[0];
  if(n % 2 == 0) sol = mod - sol;
  for(auto &i : rnd) sol = 1ll * sol * ipow(i, mod-2)%mod;
  return sol;
}
\end{cpp}

\subsection{Freivald's Algorithm}
$n \times n$ 행렬 $A,B,C$가 주어졌을 때 $AB = C$ 판별하기:\begin{enumerate}
    \item $n \times 1$ 랜덤 0/1 벡터 $r$을 만듦
    \item $A \times (Br) - Cr = (0,0,\cdots,0)^T$이면 YES, 아니면 NO.
$AB = C$이면 무조건 YES, 아니면 $\leq \frac{1}{2}$ 확률로 YES. 여러 번 반복해서 YES만 계속 나오면 높은 확률로 $AB = C$.

TODO 코드로 바꾸기
\end{enumerate}