\subsection{Hopcroft-Karp (1-indexed)}
\begin{cpp}
// 1-indexed
const int INF = 0x3f3f3f3f;
struct Hopk{
  int n, m; vector<vector<int>> adj;
  vector<int> pu, pv, dist;
  Hopk(int n, int m): n(n), m(m), adj(n+1),
    pu(n+1), pv(m+1), dist(n+1, -1) {}
  void connect(int a, int b){adj[a].push_back(b);}
  bool bfs(){
    queue<int> Q;
    for(int u=1; u<=n; u++){
      if(!pu[u]) dist[u] = 0, Q.push(u);
      else dist[u] = INF;
    }
    dist[0] = INF;
    while(!Q.empty()){
      int u = Q.front(); Q.pop();
      if(dist[u] >= dist[0]) continue;
      for(int v: adj[u]) if(dist[pv[v]] == INF){
        dist[pv[v]] = dist[u] + 1;
        Q.push(pv[v]);
      }
    }
    return dist[0] != INF;
  }
  bool dfs(int u){
    if(!u) return true;
    for(int v: adj[u])
      if(dist[pv[v]] == dist[u] + 1 && dfs(pv[v])){
        pv[v] = u, pu[u] = v; return true;
    }
    dist[u] = INF;
    return false;
  }
  int send(){
    int match = 0;
    while(bfs()) for(int u=1; u<=n; u++)
      if(!pu[u]) match+= dfs(u);
    return match;
  }
};
\end{cpp}

\subsection{Dinic}
My code is flawed, please add the correct one. TODO

\subsection{MCMF}
My code is flawed, please add the correct one. TODO

\subsection{LR-Flow}
방법 1: a번 정점에서 b번 정점으로 가는 하한 l, 상한 r인 간선이 있을 때, a번 정점에서 b번 정점으로 가는 유량 l, 비용 -1인 간선, 유량 r-l, 비용 0인 간선으로 만든 뒤 MCMF

방법 2: a번 정점에서 b번 정점으로 가는 하한 l, 상한 r인 간선이 있을 때, 새로운 source에서 b번 정점으로 가는 유량 l인 간선 추가, a번 정점에서 새로운 sink로 가는 유량 l인 간선 추가, a번 정점에서 b번 정점으로 가는 유량 r-l인 간선으로 만들고, 기존 sink에서 기존 source로 가는 유량 무한인 간선 추가 이후 최대 유량이 l의 합이 되는지 확인

\subsection{Blossom}
\begin{cpp}
// Thanks to koosaga for code
struct Blossom{
  int n, t;
  vector<vector<int>> adj;
  vector<int> orig, par, vis, match, aux;
  queue<int> Q;
  Blossom(int n): n{n}, t{0}, adj(n+1), orig(n+1),
    par(n+1), vis(n+1), match(n+1), aux(n+1) {}
  void connect(int a, int b){
    adj[a].push_back(b);
    adj[b].push_back(a);
  }

  void augment(int u, int v){
    int pv = v, nv;
    do{
      pv = par[v], nv = match[pv];
      match[v] = pv, match[pv] = v;
      v = nv;
    } while(u != pv);
  }
  int lca(int v, int w){
    ++t;
    while(1){
      if(v){
        if(aux[v] == t) return v;
        aux[v] = t, v = orig[par[match[v]]];
      }
      swap(v, w);
    }
  }
  void blossom(int v, int w, int a){
    while(orig[v] != a){
      par[v] = w, w = match[v];
      if(vis[w] == 1) Q.push(w), vis[w] = 0;
      orig[v] = orig[w] = a;
      v = par[w];
    }
  }
  bool bfs(int u){
    fill(entire(vis), -1), iota(entire(orig), 0);
    Q = queue<int>(); Q.push(u), vis[u] = 0;
    while(!Q.empty()){
      int v = Q.front(); Q.pop();
      for(int x: adj[v]){
        if(vis[x] == -1){
          par[x] = v, vis[x] = 1;
          if(!match[x]) return augment(u, x), true;
          Q.push(match[x]); vis[match[x]] = 0;
        }
        else if(vis[x] == 0 && orig[v] != orig[x]){
          int a = lca(orig[v], orig[x]);
          blossom(x, v, a), blossom(v, x, a);
        }
      }
    }
    return false;
  }

  int solve(){
    int ans = 0;
    for(int x=1; x<=n; x++) if(!match[x]){
      for(int y: adj[x]) if(!match[y]){
        match[x] = y, match[y] = x;
        ++ans; break;
      }
    }
    for(int i=1; i<=n; i++) if(!match[i] && bfs(i)) ++ans;
    return ans;
  }
};
\end{cpp}