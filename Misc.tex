\subsection{Hungarian}
\begin{cpp}
int hungarian(vector<vector<int>>& mat){
  int n = mat.size(), m = mat[0].size();
  vector<int> u(n+1), v(m+1), p(m+1), way(m+1), minv(m+1);
  vector<char> used(m+1);
  for(int i=1; i<=n; ++i) {
    int j0 = 0; p[0] = i;
    fill(entire(minv), INF);
    fill(entire(used), false);
    while(1){
      used[j0] = true;
      int i0 = p[j0], delta = INF, j1;
      for (int j=1; j<=m; ++j) if(!used[j]) {
        int cur = mat[i0-1][j-1] - u[i0] - v[j];
        if (cur < minv[j]) minv[j] = cur, way[j] = j0;
        if (minv[j] < delta) delta = minv[j], j1 = j;
      }
      for (int j=0; j<=m; ++j) {
        if (used[j]) u[p[j]] += delta, v[j] -= delta;
        else minv[j] -= delta;
      }
      j0 = j1;
      if(p[j0] == 0) break;
    }
    while(1){
      int j1 = way[j0];
      p[j0] = p[j1]; j0 = j1;
      if(!j0) break;
    }
  }
  //for (int j = 1; j <= m; ++j) matched[p[j]] = j;
  return -v[0];
}
\end{cpp}

\subsection{Lindström–Gessel–Viennot lemma}

edge-weighted DAG $G$가 있다. 출발점 집합 $A$, 도착점 집합 $B$ ($|A| = |B| = n$, $A \cap B = \emptyset$에 대해, 

\begin{itemize}
    \item $\omega(P)$: 경로 $P$에 있는 간선 가중치들의 곱
    \item $e(a, b) = \sum_{P:a \rightarrow b} \omega(P)$ (모든 $a \rightarrow b$ 경로에 대해 합쳐줌)
    \item $M_{i,j} = e(a_{i}, b_{j})$로 두면, $A$에서 $B$로 가는 $n$개의 non-intersecting path $P_{1}, \cdots, P_{n}$에 대해 ($P_i$는 $a_i$에서 $b_{\sigma(i)}$로 가는 경로)
    $$\det(M) = \sum_{(P_1, P_2, \cdots, P_n) : A \rightarrow B} sign(\sigma(P)) \prod_{i=1}^{n} \omega(P_{i})$$
\end{itemize}

용례: $\sigma$로 가능한 게 $[1, 2, \cdots, n]$뿐이라면, $\omega = 1$로 설정하고 모든 non-intersecting path의 경우의 수를 구할 수 있음.

\subsection{Convex Hull Trick}

점화식이 $D[i] = \min_{j<i}(a[j]x[i] + b[j]) + const[i]$이고, $a[j]$가 오름/내림차순이면, $O(N^2)$이 $O(NlogN)$ 또는 $O(N)$으로 줄어든다.

\begin{cpp}
// Assumption x[i] <= x[i+1], decreasing slopes
template <typename T>
struct CHT{
  vector<pair<T,T>> hull; int p=0;
  long double cr(int x){
    auto [ax,bx] = hull[x]; auto [ay,by] = hull[x+1];
    return (long double)(by-bx) / (ax-ay);
  }

  void insert(T a, T b){
    hull.push_back({a,b});
    while(hull.size() > 2 && cr(hull.size()-3) >
      cr(hull.size()-2)) hull.erase(hull.end()-2);
  }

  T get(T x){
    while(p+1 < hull.size() && cr(p) <= x) p++;
    auto [a,b] = hull[p];
    return a*x + b;
  }
};
\end{cpp}

TODO NlogN

TODO increasing

\subsection{DNC Optimization}

점화식이 $D[t][i] = \min_{k<i}(D[t-1][k] + C[k][i])$이고, $A[t][i]$를 $D[t][i]$가 최소가 되기 위한 최소 $k$라고 두었을 때 $A[t][i] \leq A[t][i+1]$이라면, $O(KN^2)$가 $O(KNlogN)$으로 줄어든다.

특수 케이스: $C[a][c]+C[b][d] \leq C[a][d]+C[b][c]$ for $a \leq b \leq c \leq d$

TODO 코드

\subsection{Knuth Optimization}

점화식이 $D[i][j] = \min_{i<k<j}(D[i][k]+D[k][j])+C[i][j]$이고, \begin{enumerate}
    \item $C[a][c]+C[b][d] \leq C[a][d]+C[b][c]$ for $a \leq b \leq c \leq d$
    \item $C[b][c] \leq C[a][d]$ for $a \leq b \leq c \leq d$
\end{enumerate}

가 모두 성립하면, $A[i][j]$를 $D[i][j]$가 최소가 되기 위한 최소 $k$라고 두었을 때 $A[i][j-1] \leq A[i][j] \leq A[i+1][j]$ 이다. $O(N^3)$이 $O(N^2)$으로 줄어든다.

\subsection{Regex}

TODO

\subsection{Fast IO}

\begin{cpp}
static char buf[1 << 19]; // size : any number >1024
static int idx = 0;
static int bytes = 0;
static inline int _read() {
  if (!bytes || idx == bytes) {
    bytes = (int)fread(buf, sizeof(buf[0]),
      sizeof(buf), stdin);
    idx = 0;
  }
  return buf[idx++];
}
static inline int _readInt() {
  int x = 0, s = 1;
  int c = _read();
  while (c <= 32) c = _read();
  if (c == '-') s = -1, c = _read();
  while (c > 32) x = 10 * x + (c - '0'), c = _read();
  if (s < 0) x = -x;
  return x;
}
\end{cpp}

Thanks to koosaga.