\subsection{String Tokenizer}
\begin{cpp}
vector<string> split(const string &s, char dm){
  // Returns a vector of strings tokenized by dm.
  stringstream ss(s);
  string item; vector<string> tokens;
  while(getline(ss,item,dm)) tokens.push_back(item);
  return tokens;
}
\end{cpp}

\subsection{KMP}
\begin{cpp}
vector<int> fail(string& w){
  vector<int> T(w.size());
  for(size_t i=1, j=0; i < w.size(); i++){
    while(j && w[i] != w[j]) j = T[j-1];
    if(w[i] == w[j]) T[i] = ++j;
  }
  return T;
}

int kmp(string& s, string& w, vector<int>& T){
  int m = 0, i = 0;
  while(m+i < (int)s.size()){
    if(w[i] == s[m+i]){if(i++==(int)w.size()-1) return m;}
    else if(i) m = m+i-T[i], i = T[i];
    else m++, i = 0;
  }
  return -1;
}
\end{cpp}

\subsection{Manacher}
\begin{cpp}
/* Insert space if you need; then
the actual length centered at i is
(M[i]+1)/2*2 if i%2 == 1; M[i]/2*2 + 1 otherwise */
vector<int> manacher(string& s){
  int n = s.size(), R = -1, p = -1;
  vector<int> A(n);
  for(int i=0; i<n; i++){
    if(i <= R) A[i] = min(A[2*p-i], R-i);
    while(i-A[i]-1 >= 0 && i+A[i]+1 < n &&
      s[i-A[i]-1] == s[i+A[i]+1]) A[i]++;
    if(i+A[i] > R) R = i+A[i], p = i;
  }
  return A;
}
\end{cpp}

\subsection{Z}
\begin{cpp}
vector<int> Z(string& s){
  int n = s.size(), L=0, R=0;
  vector<int> z(n); z[0] = n;
  for(int i=1; i<n; i++){
    if(i > R){
      L = R = i;
      while(R < n && s[R-L] == s[R]) R++;
      z[i] = R-L; R--;
      continue;
    }
    int k = i - L;
    if(z[k] < R-i+1) z[i] = z[k];
    else{
      L = i;
      while(R < n && s[R-L] == s[R]) R++;
      z[i] = R-L; R--;
    }
  }
  return z;
}
\end{cpp}

\subsection{Aho-Corasick}
\begin{cpp}
const int ALPHA = 26;
struct Node{
  bool end = false, root;
  Node *fail = nullptr, *out = nullptr;
  Node *nxt[ALPHA]; 
  Node(bool root = false): root{root}
    {fill(nxt, nxt+ALPHA, nullptr);}
};

struct Trie{
  int tonum(char c){return c - 'a';}
  
  Node* root = new Node(true);
  bool labeled = false;
  void add(string& s){
    Node* n = root;
    for(char& c: s){
      int x = tonum(c);
      if(!n->nxt[x]) n->nxt[x] = new Node();
      n = n->nxt[x];
    }
    n->end = true;
  }
  void label(){
    queue<Node*> Q; Q.push(root);
    while(!Q.empty()){
      Node *p = Q.front(); Q.pop();
      for(int i=0; i<ALPHA; i++) if(p->nxt[i]){
        Node *pp = p, *q = p->nxt[i];
        while(1){
          if(pp->root){q->fail = pp; break;}
          if(pp->fail->nxt[i])
            {q->fail = pp->fail->nxt[i]; break;}
          pp = pp->fail;
        }
        Q.push(q);
      }
      if(p->root) continue;
      else if(p->end) p->out = p;
      else if(p->fail->out) p->out = p->fail->out;
    }
  }
  int aho(string& s){
    if(!labeled) labeled = true, label();
    Node *p = root; size_t i = 0;
    while(i <= s.size()){
      if(p->out) return 1; // change this as you need
      if(i == s.size()) break;
      int c = tonum(s[i]);
      if(p->nxt[c]) p = p->nxt[c], i++;
      else if(p->root) i++;
      else p = p->fail;
    }
    return 0; // change this as you need
  }
};
\end{cpp}

\subsection{Suffix Array, LCP}
\begin{cpp}
// change string& to something else if you need
vector<int> suffixarray(string& in) {
  int n = (int)in.size(), c = 0;
  vector<int> temp(n), p2b(n), bckt(n), bpos(n), out(n);
  for(int i=0; i<n; i++) out[i] = i;
  sort(out.begin(), out.end(),
       [&](int a, int b){return in[a] < in[b];});
  for(int i=0; i<n; i++) {
    bckt[i] = c;
    if (i + 1 == n || in[out[i]] != in[out[i+1]]) c++;
  }
  for (int h = 1; h < n && c < n; h <<= 1) {
    for (int i=0; i<n; i++) p2b[out[i]] = bckt[i];
    for (int i=n-1; i>=0; i--) bpos[bckt[i]] = i;
    for (int i=0; i<n; i++) if (out[i] >= n-h)
      temp[bpos[bckt[i]]++] = out[i];
    for (int i=0; i<n; i++) if (out[i] >= h)
      temp[bpos[p2b[out[i]-h]]++] = out[i] - h;
    c = 0;
    for (int i = 0; i + 1 < n; i++) {
      int a = (bckt[i] != bckt[i+1]) || (temp[i] >= n-h)
          || (p2b[temp[i+1]+h] != p2b[temp[i]+h]);
      bckt[i] = c; c += a;
    }
    bckt[n-1] = c++;
    temp.swap(out);
  }
  return out;
}

// change string& to something else if you need
vector<int> lcparray(string& in, vector<int>& sa) {
  int n = (int)in.size();
  if (n == 0) return vector<int>();
  vector<int> rank(n), height(n - 1);
  for(int i=0; i<n; i++) rank[sa[i]] = i;
  for(int i=0, h=0; i<n; i++){
    if(rank[i] == 0) continue;
    int j = sa[rank[i]-1];
    while (i+h < n && j+h < n && in[i+h]==in[j+h]) h++;
    height[rank[i]-1] = h;
    if (h > 0) h--;
  }
  return height;
}
\end{cpp}